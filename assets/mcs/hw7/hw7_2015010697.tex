% Modified based on Xiaoming Sun's template
\documentclass{article}
\usepackage{amsmath,amsfonts,amsthm,amssymb}
\usepackage{setspace}
\usepackage{fancyhdr}
\usepackage{lastpage}
\usepackage{extramarks}
\usepackage{chngpage}
\usepackage{soul,color}
\usepackage{graphicx,float,wrapfig}

\usepackage{fontspec}
\usepackage{CJKutf8}

\usepackage{hyperref}
\hypersetup{
    colorlinks=true,
    linkcolor=blue,
    filecolor=magenta,      
    urlcolor=cyan,
}

\newcommand{\Class}{Mathematics for Computer Science}

% Homework Specific Information. Change it to your own
\newcommand{\Title}{Homework 7}

% In case you need to adjust margins:
\topmargin=-0.45in      %
\evensidemargin=0in     %
\oddsidemargin=0in      %
\textwidth=6.5in        %
\textheight=9.0in       %
\headsep=0.25in         %

% Setup the header and footer
\pagestyle{fancy}                                                       %
\chead{\Title}  %
\rhead{\firstxmark}                                                     %
\lfoot{\lastxmark}                                                      %
\cfoot{}                                                                %
\rfoot{Page\ \thepage\ of\ \protect\pageref{LastPage}}                          %
\renewcommand\headrulewidth{0.4pt}                                      %
\renewcommand\footrulewidth{0.4pt}                                      %

%%%%%%%%%%%%%%%%%%%%%%%%%%%%%%%%%%%%%%%%%%%%%%%%%%%%%%%%%%%%%
% Some tools
\newcommand{\enterProblemHeader}[1]{\nobreak\extramarks{#1}{#1 continued on next page\ldots}\nobreak%
                                    \nobreak\extramarks{#1 (continued)}{#1 continued on next page\ldots}\nobreak}%
\newcommand{\exitProblemHeader}[1]{\nobreak\extramarks{#1 (continued)}{#1 continued on next page\ldots}\nobreak%
                                   \nobreak\extramarks{#1}{}\nobreak}%

\newcommand{\homeworkProblemName}{}%
\newcounter{homeworkProblemCounter}%
\newenvironment{homeworkProblem}[1][Problem \arabic{homeworkProblemCounter}]%
  {\stepcounter{homeworkProblemCounter}%
   \renewcommand{\homeworkProblemName}{#1}%
   \section*{\homeworkProblemName}%
   \enterProblemHeader{\homeworkProblemName}}%
  {\exitProblemHeader{\homeworkProblemName}}%

\newcommand{\homeworkSectionName}{}%
\newlength{\homeworkSectionLabelLength}{}%
\newenvironment{homeworkSection}[1]%
  {% We put this space here to make sure we're not connected to the above.

   \renewcommand{\homeworkSectionName}{#1}%
   \settowidth{\homeworkSectionLabelLength}{\homeworkSectionName}%
   \addtolength{\homeworkSectionLabelLength}{0.25in}%
   \changetext{}{-\homeworkSectionLabelLength}{}{}{}%
   \subsection*{\homeworkSectionName}%
   \enterProblemHeader{\homeworkProblemName\ [\homeworkSectionName]}}%
  {\enterProblemHeader{\homeworkProblemName}%

   % We put the blank space above in order to make sure this margin
   % change doesn't happen too soon.
   \changetext{}{+\homeworkSectionLabelLength}{}{}{}}%

\newcommand{\Answer}{\ \\\textbf{Answer:} }
\newcommand{\Acknowledgement}[1]{\ \\{\bf Acknowledgement:} #1}

%%%%%%%%%%%%%%%%%%%%%%%%%%%%%%%%%%%%%%%%%%%%%%%%%%%%%%%%%%%%%


%%%%%%%%%%%%%%%%%%%%%%%%%%%%%%%%%%%%%%%%%%%%%%%%%%%%%%%%%%%%%
% Make title
\title{\textmd{\bf \Class: \Title}}
\date{April 15, 2019}
\author{Xingyu Su 2015010697}
%%%%%%%%%%%%%%%%%%%%%%%%%%%%%%%%%%%%%%%%%%%%%%%%%%%%%%%%%%%%%

\begin{document}
\begin{spacing}{1.1}
\maketitle \thispagestyle{empty}

%%%%%%%%%%%%%%%%%%%%%%%%%%%%%%%%%%%%%%%%%%%%%%%%%%%%%%%%%%%%%
% Begin edit from here

%%%%%%%%%%%%%%%%%%%%%%%%%%%%%%%%%%%%%%%%%%%%%%%%%%%%%%%%%%%%%
\begin{homeworkProblem}[Special Problem 1]
\hspace{1em}
In the attachment to this homework set, we give a summary of the lecture delivered today. In particular, Theorem 1 states that, under some assumptions on $f(z)$, the power series coefficients $a_n$ of $f(z)$ can be expressed as a linear combination of residues of $f(z)/z_{n+1}$ at $z = z_j$. Assume that we have proved Theorem 1.

\textbf{Question:} Prove Corollary 2.

\Answer

From 5.1 of \href{http://202.116.32.252:8080/fbhs/upload/file/20161120/20161120180759_71344.pdf}{\begin{CJK}{UTF8}{gbsn}《复变函数简明教程》\end{CJK}}, we know for $z_0 \neq \infty$  is a m order pole singularity of $f(z)$, residue of $f(z)$ at $z=z_0$ is:
\begin{equation*}
Res(f,z_0) = \frac{1}{(m-1)!}\lim_{z \to z_0}{\frac{d^{m-1}}{dz^{m-1}}[(z-z_0)^m f(z)]}.
\end{equation*}

So with Theorem 1, power expansion of $f$ at $z=0$, $f(z)=\sum_{n\geq0}a_nz^n$ having (Since the power expansion is from $z=0$)
\begin{align*}
a_n
  &= - \sum_{i\geq1}Res(\frac{f(z)}{z^{n+1}},z_i) \\
  &= - \sum_{i\geq1}\frac{1}{(m-1)!}\lim_{z \to z_i}{\frac{d^{m-1}}{dz^{m-1}}[(z-z_i)^m \frac{f(z)}{z^{n+1}}]} \\
  &= - \sum_{i\geq1}\frac{1}{z_i^{n+1}}\frac{1}{(m-1)!}\lim_{z \to z_i}{\frac{d^{m-1}}{dz^{m-1}}[(z-z_i)^m f(z)]} \\
  &= - \sum_{i\geq1}\frac{Res(f, z_i)}{z_i^{n+1}}
\end{align*}

\end{homeworkProblem}
%%%%%%%%%%%%%%%%%%%%%%%%%%%%%%%%%%%%%%%%%%%%%%%%%%%%%%%%%%%%%

%%%%%%%%%%%%%%%%%%%%%%%%%%%%%%%%%%%%%%%%%%%%%%%%%%%%%%%%%%%%%
\begin{homeworkProblem}[Special Problem 2]
\hspace{1em}
Let
\begin{equation*}
f(z) =\frac{3}{(z - 5i)^2}.
\end{equation*}

  (a) Let $n > 0$. Determine the residue of $f(z)/z^{n+1}$ at $z = 5i$.

  (b) Use Theorem 1 to determine the power series expansion of $f(z) = \sum_{n\geq 0} a_n z^n$. Does this agree with the answer you would get if you apply Newton’s binomial theorem to the expression $(1 - z/5i)^{-2}$?

\Answer

  (a) Easy to know that $z_1=5i$ is a second order pole singularity of $f$. So $f(z)/z^{n+1}$ have a second order pole singularity at $z_1=5i$ expect $z_0=0$. With the formula mentioned above, we have:
\begin{align*}
Res(\frac{f(z)}{z^{n+1}},z_1)
  &= \frac{1}{1!}\lim_{z \to 5i}{\frac{d}{dz}[(z-5i)^2 \frac{f(z)}{z^{n+1}}]} \\
  &= 3\lim_{z \to 5i}{\frac{d}{dz}[\frac{1}{z^{n+1}}]} \\
  &= -\frac{3}{25} \frac{(n+1)}{(5i)^{n}}
\end{align*}

  (b) With Theorem 1 and results from (a), we have:
\begin{equation*}
a_n = -\sum_{i\geq1} Res(\frac{f(z)}{z^{n+1}},z_i) = -Res(\frac{f(z)}{z^{n+1}},z_1) = -\frac{3}{25} \frac{n+1}{(5i)^{n}}
\end{equation*}

And Newton's binomial theorem with negative exponents have:
\begin{equation*}
(1+z)^{-k} = \sum_{n\geq0}(-1)^n\binom{k+n-1}{n}z^{n}
\end{equation*}

So there is:
\begin{align*}
(1-\frac{z}{5i})^{-2}
  &= \sum_{n\geq0}(-1)^n\binom{n+1}{n}\left(-\frac{z}{5i}\right)^{n} \\
  &= \sum_{n\geq0}\binom{n+1}{n}(5i)^{-n}z^n
\end{align*}

And known that:
\begin{equation*}
f(z) = \frac{3}{(z-5i)^2}=\frac{3}{(5i)^2}(\frac{5i-z}{5i})^{-2} = -\frac{3}{25}(1-\frac{z}{5i})^{-2}
\end{equation*}

So power series expansion coefficients $a'_n$ from Newton's binomial theorem is:
\begin{equation*}
a'_n = -\frac{3}{25} \binom{n+1}{n}(5i)^{-n} = -\frac{3}{25}\frac{n+1}{(5i)^n} = a_n
\end{equation*}

\end{homeworkProblem}
%%%%%%%%%%%%%%%%%%%%%%%%%%%%%%%%%%%%%%%%%%%%%%%%%%%%%%%%%%%%%

%%%%%%%%%%%%%%%%%%%%%%%%%%%%%%%%%%%%%%%%%%%%%%%%%%%%%%%%%%%%%
\begin{homeworkProblem}[Special Problem 3]
\hspace{1em}
Let $A(z) = \frac{1}{\lambda-e^z}$ be a function over the complex plane, where $\lambda > 1$ is a real number.

(a) Where are all the singularities of $A$ on the complex plane? Are they isolated singularities? Determine the residue of $A$ at each of its isolated singularities.

(b) Consider the power series expansion $A(z) = \sum_{n\geq0} a_n z^n$ in the neighborhood of $z = 0$. Find a closed-form expression $g(n)$ in variable $n$, such that $\lim_{n\to\infty} \frac{a_n}{g(n)} = 1$. You should give your reasoning rigorously.

(c) Consider the following recurrence relation: $b_0 = 1$, and for $n \geq 1$,
\begin{equation*}
b_n = \sum_{0\leq k \leq n-1} b_k \binom{2n}{2k}.
\end{equation*}
Find a closed-form expression $h(n)$ in variable $n$, such that $\lim_{n\to\infty}\frac{b_n}{h(n)} = 1$.
\Answer

(a) Easy to know that when $z=\ln \lambda+2k\pi$ for $k \in \mathcal{Z}$, $\lambda-e^z=0$, so $z_k = \ln \lambda + 2 k\pi, k \in \mathcal{Z}$ are isolated singularities of $A$.
\begin{align*}
Res(f,\ln \lambda + 2k\pi) = Res(f,\ln \lambda) 
  &= \lim_{z \to \ln \lambda}(z-\ln \lambda)\frac{1}{\lambda-e^z} = \lim_{z \to \ln \lambda}\frac{1}{-e^z} = -\frac{1}{\lambda}
\end{align*}

(b) Since $A(z) = \sum_{n\geq0}a_n z^n$ and all singularities $z_k$ have only one order. So $a_n$:
\begin{align*}
a_n 
  &= -\sum_{k=-\infty}^{+\infty}Res(\frac{1}{(\lambda - e^z)z^{n+1}}, z_k) \\
  &= -\sum_{k=-\infty}^{+\infty} \lim_{z\to z_k} (z-z_k)\frac{1}{\lambda-e^z}\frac{1}{z^{n+1}} \\
  &= \frac{1}{\lambda}\sum_{k=-\infty}^{+\infty}\frac{1}{z_k^{n+1}}\\
  &= \frac{1}{\lambda}\sum_{k=-\infty}^{+\infty}\frac{1}{(\ln\lambda+2k\pi)^{n+1}}
\end{align*}

And obviously, when $n\to\infty$, terms smaller than 1 with exponent $n+1$ decreases rapidly. So with $g(n) = \frac{1}{\lambda}\frac{1}{(\ln \lambda+2m\pi)^{n+1}}$ where:
\begin{equation*}
|\ln \lambda + 2m\pi| = \min_{k}\{|\ln \lambda + 2k\pi|\}
\end{equation*}

For example, when $\lambda=e, \ln\lambda=1$, $m=-1$, $g(n) = (1-2\pi)^{-(n+1)}/e$

(c) With
\begin{align*}
b_n &= \sum_{k=0}^{n-1} b_k \binom{2n}{2k} \\
    &= \sum_{k=0}^{n-2} b_k \binom{2n}{2k} + n(2n-1) b_{n-1} \\
    &\leq \left[\frac{(2n-1)n}{6}+n(2n-1)\right]b_{n-1} \\
    &= \frac{7}{6}(2n^2-n)b_{n-1}
\end{align*}

And
\begin{align*}
b_n &= \sum_{k=0}^{n-1} b_k \binom{2n}{2k} \\
    &= \sum_{k=0}^{n-2} b_k \binom{2n}{2k} + n(2n-1) b_{n-1} \\
    &\geq (2n^2-n)b_{n-1}
\end{align*}
So $h(n) = (2n-1)!!n!$ is a close form that has $\lim_{n\to\infty}\frac{b_n}{h(n)} = 1$
\end{homeworkProblem}
%%%%%%%%%%%%%%%%%%%%%%%%%%%%%%%%%%%%%%%%%%%%%%%%%%%%%%%%%%%%%

% ACKNOWLEGEMENT
\Acknowledgement

Thanks to \href{http://202.116.32.252:8080/fbhs/upload/file/20161120/20161120180759_71344.pdf}{\begin{CJK}{UTF8}{gbsn}《复变函数简明教程》\end{CJK}} for SP1


% End edit to here
%%%%%%%%%%%%%%%%%%%%%%%%%%%%%%%%%%%%%%%%%%%%%%%%%%%%%%%%%%%%%

\end{spacing}
\end{document}

%%%%%%%%%%%%%%%%%%%%%%%%%%%%%%%%%%%%%%%%%%%%%%%%%%%%%%%%%%%%%
