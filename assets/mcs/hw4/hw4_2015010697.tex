% Modified based on Xiaoming Sun's template
\documentclass{article}
\usepackage{amsmath,amsfonts,amsthm,amssymb}
\usepackage{setspace}
\usepackage{fancyhdr}
\usepackage{lastpage}
\usepackage{extramarks}
\usepackage{chngpage}
\usepackage{soul,color}
\usepackage{graphicx,float,wrapfig}

\usepackage{hyperref}
\hypersetup{
    colorlinks=true,
    linkcolor=blue,
    filecolor=magenta,      
    urlcolor=cyan,
}

\newcommand{\Class}{Mathesmatics for Computer Science}

% Homework Specific Information. Change it to your own
\newcommand{\Title}{Homework 4}

% In case you need to adjust margins:
\topmargin=-0.45in      %
\evensidemargin=0in     %
\oddsidemargin=0in      %
\textwidth=6.5in        %
\textheight=9.0in       %
\headsep=0.25in         %

% Setup the header and footer
\pagestyle{fancy}                                                       %
\chead{\Title}  %
\rhead{\firstxmark}                                                     %
\lfoot{\lastxmark}                                                      %
\cfoot{}                                                                %
\rfoot{Page\ \thepage\ of\ \protect\pageref{LastPage}}                          %
\renewcommand\headrulewidth{0.4pt}                                      %
\renewcommand\footrulewidth{0.4pt}                                      %

%%%%%%%%%%%%%%%%%%%%%%%%%%%%%%%%%%%%%%%%%%%%%%%%%%%%%%%%%%%%%
% Some tools
\newcommand{\enterProblemHeader}[1]{\nobreak\extramarks{#1}{#1 continued on next page\ldots}\nobreak%
                                    \nobreak\extramarks{#1 (continued)}{#1 continued on next page\ldots}\nobreak}%
\newcommand{\exitProblemHeader}[1]{\nobreak\extramarks{#1 (continued)}{#1 continued on next page\ldots}\nobreak%
                                   \nobreak\extramarks{#1}{}\nobreak}%

\newcommand{\homeworkProblemName}{}%
\newcounter{homeworkProblemCounter}%
\newenvironment{homeworkProblem}[1][Problem \arabic{homeworkProblemCounter}]%
  {\stepcounter{homeworkProblemCounter}%
   \renewcommand{\homeworkProblemName}{#1}%
   \section*{\homeworkProblemName}%
   \enterProblemHeader{\homeworkProblemName}}%
  {\exitProblemHeader{\homeworkProblemName}}%

\newcommand{\homeworkSectionName}{}%
\newlength{\homeworkSectionLabelLength}{}%
\newenvironment{homeworkSection}[1]%
  {% We put this space here to make sure we're not connected to the above.

   \renewcommand{\homeworkSectionName}{#1}%
   \settowidth{\homeworkSectionLabelLength}{\homeworkSectionName}%
   \addtolength{\homeworkSectionLabelLength}{0.25in}%
   \changetext{}{-\homeworkSectionLabelLength}{}{}{}%
   \subsection*{\homeworkSectionName}%
   \enterProblemHeader{\homeworkProblemName\ [\homeworkSectionName]}}%
  {\enterProblemHeader{\homeworkProblemName}%

   % We put the blank space above in order to make sure this margin
   % change doesn't happen too soon.
   \changetext{}{+\homeworkSectionLabelLength}{}{}{}}%

\newcommand{\Answer}{\ \\\textbf{Answer:} }
\newcommand{\Acknowledgement}[1]{\ \\{\bf Acknowledgement:} #1}

%%%%%%%%%%%%%%%%%%%%%%%%%%%%%%%%%%%%%%%%%%%%%%%%%%%%%%%%%%%%%


%%%%%%%%%%%%%%%%%%%%%%%%%%%%%%%%%%%%%%%%%%%%%%%%%%%%%%%%%%%%%
% Make title
\title{\textmd{\bf \Class: \Title}}
\date{March 25, 2019}
\author{Xingyu Su 2015010697}
%%%%%%%%%%%%%%%%%%%%%%%%%%%%%%%%%%%%%%%%%%%%%%%%%%%%%%%%%%%%%

\begin{document}
\begin{spacing}{1.1}
\maketitle \thispagestyle{empty}

%%%%%%%%%%%%%%%%%%%%%%%%%%%%%%%%%%%%%%%%%%%%%%%%%%%%%%%%%%%%%
% Begin edit from here

%%%%%%%%%%%%%%%%%%%%%%%%%%%%%%%%%%%%%%%%%%%%%%%%%%%%%%%%%%%%%
\begin{homeworkProblem}[LPV 6.10.22]
We are given $n + 1$ numbers from the set $\{1, 2, \cdots , 2n\}$. Prove that there are two numbers among them such that one divides the other.

\Answer 

Divide each of the numbers $a_i$ into groups $A_k$ by $a_i=2^{p}\cdot k$ with bigest integer $p$, which is:
\begin{equation*}
A_k = {2^0 k + 2^1 k+2^2 k+\cdots}
\end{equation*}

Obviously every $k$ is $k\equiv 1(mod2)$, and we have a n-segmentation of $\{1, 2, \cdots, 2n \}$:
\begin{align*}
A_{1}&={2^0\cdot1,2^1\cdot1,2^2\cdot1,\cdots};\\
A_{3}&={2^0\cdot3,2^1\cdot3,2^2\cdot3,\cdots};\\
\cdots \\
A_{2n-1}&={2^0\cdot(2n-1)};
\end{align*}

And every two numbers in each group  will have one can be divided by the other. So if we are given $n+1$ numbers, at least 2 are from one group, with pigeon hole principle known. So there are two numbers among them such that one divides the other.
\end{homeworkProblem}
%%%%%%%%%%%%%%%%%%%%%%%%%%%%%%%%%%%%%%%%%%%%%%%%%%%%%%%%%%%%%


%%%%%%%%%%%%%%%%%%%%%%%%%%%%%%%%%%%%%%%%%%%%%%%%%%%%%%%%%%%%%
\begin{homeworkProblem}[LPV 6.10.23]
What is the number of positive integers not larger than 210 and not divisible by 2, 3 or 7?

\Answer 
Similar to \textbf{6.9.1}, we have
\begin{equation*}
210-(\frac{210}{2}+\frac{210}{3}+\frac{210}{7})+(\frac{210}{2\cdot3}+\frac{210}{2\cdot7}+\frac{210}{3\cdot7})-\frac{210}{2\cdot3\cdot7} = 60
\end{equation*}
integers not larger than 210 and not divisible by 2, 3 or 7.

\end{homeworkProblem}
%%%%%%%%%%%%%%%%%%%%%%%%%%%%%%%%%%%%%%%%%%%%%%%%%%%%%%%%%%%%%


%%%%%%%%%%%%%%%%%%%%%%%%%%%%%%%%%%%%%%%%%%%%%%%%%%%%%%%%%%%%%
\begin{homeworkProblem}[Special Problem 3]
	Let $X_1 , X_2 , \cdots , X_n$ be independent Poisson trials such that $Pr\{X_i = 1\} = p_i$ . Let $X = \sum_{1\leq i\leq n} X_i$ and $\mu = E(X)$. In class we derived one version of the Chernoff Bounds regarding the probability that $X > (1 + \sigma)\mu$. Here you are asked to prove the following bounds in a similar way:

(a) For $0 <\delta < 1$,
\begin{equation*}
Pr\{X\leq(1-\delta)\mu\}\leq\left(\frac{e^{-\delta}}{(1-\delta)^{1-\delta}}\right)^\mu.
\end{equation*}

(b) Assume that $p_i = 1/2$ for all $i$. Prove the stronger bound that
\begin{equation*}
Pr{|X -\frac{n}{2}| > a} \leq 2e^{\frac{-2a^2}{n}}.
\end{equation*}

(Hint: First show that $e^t + 1 \leq 2e^{t/2+t^2/8}$ for all $t > 0$.)

\Answer 

(a) Known $Pr(X_i=1)=p_i$ and $Pr(X\leq a) \leq e^{ta} \Pi_i E[e^{-tX_i}]$ with $t>0$.
\begin{align*}
Pr(X\leq(1-\delta)\mu)
  & \leq \frac{\Pi_{i=1}^n E[e^{-tX_i}]}{e^{-t(1-\delta)\mu}}\\
  & = \frac{\Pi_{i=1}^n{[p_ie^{-t}+(1-p_i)]}}{e^{-t(1-\delta)\mu}}
\end{align*}

\hspace{1em}
And with know $1+x \leq e^x$, we have $p_i e^{-t} +(1-p_i) = p_i(e^{-t}-1)+1\leq e^{p^i(e^{-t}-1)}$. So
\begin{align*}
Pr(X\leq (1-\delta)\mu)
  &\leq \frac{\Pi_{i=1}^n e^{p_i(e^{-t}-1)}}{e^{-t(1-\delta)\mu}} \\
  &= \frac{e^{(e^{-t}-1)\sum_{i=1}^n p_i}}{e^{-t(1-\delta)\mu}} \\
  &= \frac{e^{(e^{-t}-1)\mu}}{e^{-t(1-\delta)\mu}}
\end{align*}

\hspace{1em}
Set $t=-\ln(1-\delta)$ and $t>0$ when $0<\delta<1$, then
\begin{align*}
Pr(X\leq(1-\delta)\mu) 
  &\leq \frac{e^{(e^{-t}-1)\mu}}{e^{-t(1-\delta)\mu}} \\
  &= \frac{e^{-\delta\mu}}{(1-\delta)^{(1-\delta)\mu}} \\
  &= \left[\frac{e^-\delta}{(1-\delta)^{(1-\delta)}}\right]^\mu
\end{align*}

(b) I have no idea from $e^t-1\leq 2e^{t/2+t^2/8}$. But a different prove from \textbf{Probability and Computing: Randomized Algorithms and Probabilistic Analysis} is found as below:

\hspace{1em}
Let $Y_1, Y_2, \cdots, Y_n$ be independent random variables with $Pr(Y_i=1)=Pr(Y_i=-1)=\frac{1}{2}$ and $Y=\sum_{i=1}^n Y_i$, for any $t>0$,
\begin{align*}
E[e^{tY_i}] 
  &= \frac{1}{2}e^t+\frac{1}{2}e^{-t}\\
  &= \frac{1}{2}(1+t+\frac{t^2}{2!}+\frac{t^3}{3!}+\cdots)-\frac{1}{2}(1-t+\frac{t^2}{2!}-\frac{t^3}{3!}+\cdots) \\
  &= \sum_{i\geq0}\frac{t^{2i}}{(2i)!} \\
  &= \sum_{i\geq0}\frac{(t^2/2)^i}{i!}
  &= e^{t^2/2}.
\end{align*}

\hspace{1em}
With $t=\frac{a}{n}>0$, we get
\begin{align*}
Pr(Y\geq a) 
  & \leq \frac{E[e^{tY}]}{e^{ta}} \\
  &= \frac{\Pi_{i=1}^n E[e^{tY_i}]}{e^{ta}}\\
  &= e^{n t^2/2-ta} \\
  &= e^{-{a^2}/{2n}}
\end{align*}

\hspace{1em}
So with $X_i$ have $Pr(X_i=1)=Pr(X_i=0)=\frac{1}{2}$ and $X=\sum_{i=1}^n X_i$. We get $\mu=\frac{n}{2}$ and $X=\frac{1}{2}\sum_{i=1}^n(X_i+1)=\frac{1}{2}Y+\mu$

\begin{equation*}
Pr(X\geq \mu+a)=Pr(Y\geq 2a)\leq e^{-4a^2/2n}
\end{equation*}

\hspace{1em}
with symmetry, we finally get

\begin{equation*}
Pr(|X-\frac{n}{2}|\geq a) \leq 2e^{\frac{-2a^2}{n}}
\end{equation*}

\end{homeworkProblem}


%%%%%%%%%%%%%%%%%%%%%%%%%%%%%%%%%%%%%%%%%%%%%%%%%%%%%%%%%%%%%
\begin{homeworkProblem}[Special Problem 4]
Use the Chernoff Bounds derived in class and in the above problem to prove the following inequalities: For all $0 < \delta \leq 1$

  (a) $Pr\{X \geq (1 + \delta)\mu\} \leq e^{-\mu\delta^2/3}.$

  (b) $Pr\{X \leq (1 - \delta)\mu\} \leq e^{-\mu\delta^2/2}$

\textbf{Remark} Note that it follows from (a) and (b) that $Pr\{|X - E(X)| > a\} \leq 2e^{-a^2/3E(X)}$ for all $0 < a \leq E(X)$.

\Answer 

(a) From SP3a we get a symmetry formula:
\begin{equation*}
Pr(X\geq(1+\delta)\mu) \leq \left[\frac{e^\delta}{(1+\delta)^{(1+\delta)}}\right]^\mu
\end{equation*}

\hspace{1em}
To get 
\begin{equation*}
Pr(X \geq (1 + \delta)\mu) \leq e^{-\mu\delta^2/3}
\end{equation*}

\hspace{1em}
We can get 
\begin{equation*}
\frac{e^\delta}{(1+\delta)^{(1+\delta)}}  \leq e^{-\delta^2/3}
\end{equation*}

\hspace{1em}
first.

\hspace{1em}
The derivative of upper inequality is writen as blow:
\begin{equation*}
f(\delta) = \delta-(1+\delta)\ln(1+\delta)+\frac{\delta^2}{3}\leq 0
\end{equation*}

\hspace{1em}
So 
\begin{align*}
f'(\delta) &= - \ln(1+\delta) +\frac{2}{3}\delta \\
f''(\delta)&=-\frac{1}{1+\delta}+\frac{2}{3}
\end{align*}

\hspace{1em}
So $f''(\delta)<0$ for $0<\delta<\frac{1}{2}$ and $f''(\delta)>0$ for $\frac{1}{2}<\delta\leq 1$. And $f'(0) = 0$, $f'(1)=-\ln(2)+\frac{2}{3}<0$, so $f'(\delta)<0$ for all $0<\delta\leq1$.

\hspace{1em}
With $f(0) = 0$, we are convienced now that $f(\delta)<0$ for all $0<\delta\leq1$, which equals to (a).

(b) Similary to (a), we get the derivative as below:
\begin{equation*}
g(\delta) = -\delta-(1-\delta)\ln(1-\delta)+\frac{\delta^2}{2}\leq 0
\end{equation*}

\hspace{1em}
And
\begin{align*}
g'(\delta) &=  \ln(1-\delta)+\delta\\
g''(\delta) &=  1-\frac{1}{1-\delta}
\end{align*}

\hspace{1em}
Obviously, $g''(\delta)<0$ for all $0<\delta\leq1$. And $g'(0)=0$, so $g'(\delta)<0$; $g(0)=0$, so $g(\delta)<0$ is got easily.

\end{homeworkProblem}
%%%%%%%%%%%%%%%%%%%%%%%%%%%%%%%%%%%%%%%%%%%%%%%%%%%%%%%%%%%%%


%%%%%%%%%%%%%%%%%%%%%%%%%%%%%%%%%%%%%%%%%%%%%%%%%%%%%%%%%%%%%
% ACKNOWLEGEMENT
\Acknowledgement
  
  SP3a: Wikipedia \href{https://en.wikipedia.org/wiki/Chernoff_bound#cite_note-MitzenmacherUpfal-3}{Chernoff bound}
  
  SP3b: Part 4.8 and 4.9 from book \href{https://books.google.com/books?id=0bAYl6d7hvkC&pg=PR7&hl=zh-CN&source=gbs_selected_pages&cad=3#v=onepage&q=chernoff%20bound&f=false}{Probability and Computing: Randomized Algorithms and Probabilistic Analysis}
  
% End edit to here
%%%%%%%%%%%%%%%%%%%%%%%%%%%%%%%%%%%%%%%%%%%%%%%%%%%%%%%%%%%%%

\end{spacing}
\end{document}

%%%%%%%%%%%%%%%%%%%%%%%%%%%%%%%%%%%%%%%%%%%%%%%%%%%%%%%%%%%%%
